%%%%%%%%%%%%%%%%%%%%%%%%%%%%%%%%%%%%%%%%%%%%%%%%%%%%%%%%%%%%%%%%%
% Contents: The todo chapter
% $Id: grisbi-manuel-todo.tex, v 0.4 2002/10/27 Daniel Cartron
% $Id: grisbi-manuel-todo.tex, v 0.6.0 2011/11/17 Jean-Luc Duflot
% $Id: grisbi-manuel-todo.tex, v 0.8.8 2011/XX/XX Jean-Luc Duflot
%%%%%%%%%%%%%%%%%%%%%%%%%%%%%%%%%%%%%%%%%%%%%%%%%%%%%%%%%%%%%%%%%

% benj avec cedric : modifier ce fichier s'il vous plait. Loic

\chapter{Feuille de route\label{todo} }

Grisbi est un logiciel en évolution. Nous ajoutons des fonctionnalités en permanence.  Les fonctionnalités que nous désirons ajouter dans le futur sont détaillées dans la feuille de route qui suit.

Bien évidemment cette liste est autant une indication de ce que nous voulons faire qu'un appel à contribution.  N'hésitez pas à apporter votre coopération si Grisbi vous convient et encore plus s'il ne vous convient pas.


\section{Version 0.6}

\begin{itemize}
	\item gestion des budgets: possibilité de créer autant de budgets prévisionnels 	que d'exercices, basés soit sur les catégories soit sur les imputations budgétaires;	
	\item amélioration des listes d'opérations et du formulaire;	
	\item aide contextuelle;	
	\item astuces du jour;	
	\item mélioration de l'interface;	
	\item Import CSV et Gnucash.
	%\item Catégories fiscalisables et non fiscalisables: permettra de créer un état fiscal des revenus imposables et des déductions possibles;
\end{itemize}


\section{Version 0.7}

Il s'agit d'une version de numéro impair, donc une version de développement.

\begin{itemize}
	\item gestion des prêts: simulation de tous les prêts bancaires existants, impression de tableaux d'amortissement, automatisation des opérations de remboursement;	
	\item augmentation des critères d'affichage et de tri: par numéro de chèque,
	par numéro de rapprochement, par date croissante ou décroissante, etc., et ce dans tous les onglets.
\end{itemize}


\section{version 0.8}

\begin{itemize}
	\item ajout de graphiques basés sur les états, et possibilité de les imprimer et les exporter;	
	\item feuilles de style pour les états et graphiques: possibilité de personnaliser la mise en page (polices, couleurs, logos, etc.); possibilité d'exporter et d'importer ces feuilles de style; création d'une banque de styles sur le site internet;	
	\item barres d'outils: modification des barres en fonction du contexte, possibilité de les personnaliser.
	\item développement d'un langage de macros (python).
\end{itemize}

\section{version 0.9}

Il s'agit d'une version de numéro impair, donc une version de développement.


\section{version 1.0}

Voir le chapitre Introduction, section 2.3.4 : Nouveautés de la version 1.0


\section{version 2.0}

\begin{itemize}
	\item gestion des placements: portefeuille boursier;
	\item accès aux sites boursiers par internet et mise à jour automatique des 	informations.
\end{itemize}


\section{hors planning}

\subsection{traduction dans d'autres langues que l'anglais}

Opération non planifiable puisque dépendant de l'implication de nouveaux traducteurs.
