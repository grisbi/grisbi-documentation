%%%%%%%%%%%%%%%%%%%%%%%%%%%%%%%%%%%%%%%%%%%%%%%%%%%%%%%%%%%%%%%%%
% Contents: The title page
% $Id: grisbi-manuel-title.tex,v 0.4 2002/10/27 Daniel Cartron
%%%%%%%%%%%%%%%%%%%%%%%%%%%%%%%%%%%%%%%%%%%%%%%%%%%%%%%%%%%%%%%%%

%\begin{titlepage}

%\pagecolor{jaunegrisbi}

%\thispagestyle{empty}

%\title{
%\begin{center}
%\textcolor{bleugrisbi}{
%	\fontsize{28mm}{0mm}\selectfont
%	Manuel\\
%	\fontsize{10mm}{10mm}\selectfont
%	de\\
%	\fontsize{18mm}{18mm}\selectfont
%	l'utilisateur\\
%	\fontsize{10mm}{10mm}\selectfont
%	de\\
%	\fontsize{32mm}{10mm}\selectfont
%	\textsc{ Grisbi}
%	}
%\begin{figure}[htbp]
%\begin{center}
%\includegraphics{image/screenshot/logo_grisbi_grand}
%\end{center}
%\end{figure}
%\end{center}
%}

%\author{\textcolor{ocregrisbi}{Copyright © 2001-2002 Daniel \familyname{Cartron}}}

%\date{\textcolor{vertgrisbi}{\small{Version 0.3.3 du 15 octobre 2002}}}

%\end{titlepage}

%\maketitle

%\newpage

%\thispagestyle{empty}

%\vspace*{8cm}

%\textbf{

%Grisbi est un logiciel de comptabilité personnelle pour Linux.

%Son utilisation est particulièrement intuitive, à tel point que la lecture de ce manuel est presqu'inutile pour un usage courant \dots 

%Par contre Grisbi intègre également des fonctionnalités avancées très performantes accessibles par les nombreuses options de configuration et de personnalisation. Et là il devient très intéressant de lire attentivement ce manuel, pour pouvoir tirer parti de toute la puissance du logiciel.

%Grisbi est ainsi tout à fait adapté pour la comptabilité de petites associations, utilisation facilitée par le \emph{Guide de la comptabilité d'associations} dédié à ce logiciel. 
%}

%\vspace*{2cm}

%\begin{flushleft}

%\begin{figure}[htbp]
%\begin{flushleft}
%\includegraphics{image/screenshot/logo_grisbi_petit}
%\end{flushleft}
%\end{figure}

%\textbf{www.grisbi.org}

%\end{flushleft}
\maketitle

%\tableofcontents
%\newpage

%\ifIllustration
%\listoffigures
%\newpage
%\else
%\fi
