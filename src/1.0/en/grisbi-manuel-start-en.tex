%%%%%%%%%%%%%%%%%%%%%%%%%%%%%%%%%%%%%%%%%%%%%%%%%%%%%%%%%%%%%%%%%
% Contents : The first start chapter
% $Id : grisbi-manuel-start.tex, v 0.4 2002/10/27 Daniel Cartron
% $Id : grisbi-manuel-start.tex, v 0.5.0 2004/06/01 Loic Breilloux
% $Id : grisbi-manuel-start.tex, v 0.6.0 2011/11/17 Jean-Luc Duflot
% $Id : grisbi-manuel-start.tex, v 0.8.9 2012/04/27 Jean-Luc Duflot
% $Id : grisbi-manuel-start.tex, v 1.0 2014/02/12 Jean-Luc Duflot
%%%%%%%%%%%%%%%%%%%%%%%%%%%%%%%%%%%%%%%%%%%%%%%%%%%%%%%%%%%%%%%%%


\chapter{Initial set-up of Grisbi\label{start}}


\section{Initial Set-up Wizard\label{start-first}}


When Grisbi first starts, the Initial Set-up Wizard is displayed to
help you configure the application. It consists of two stages, which concern
management of the  \indexword{account file}\index{account file} (automatic loading and saving,
encryption and backups).
It is advisable to check the options:

\begin{itemize}
 \item automatic loading of the last used file  ;
 \item automatic save on close ;
 \item make a backup copy before saving the file.
\end{itemize}

This wizard is automatically followed by a second wizard, for creating the \indexword{account file}\index{account file} . Then a third assistant starts, the account creation wizard, to create the first account. All of this is described in detail in section \ref{start-newfile}  below.

At any time you can exit any wizard with the \menu{Cancel} button.

If you do not want to use the Set-up Wizard, you can open a sample file instead (see the next section).

\section{Sample file\label{start-example}}
\strong{Translators Note:} The only sample files currently available are in French only, these can now also be found at \url{https://github.com/LudovicRousseau/grisbi-examples}.

If you want to use Grisbi immediately without having to go through the full set-up, for example to get an idea of the possibilities of this program, you can download the  \file{Example\_1.0.gsb} file, either from Grisbi's website in the Download section{\siteGrisbiTelechargement}, or on the {Sourceforge}{\siteSourceForgeDocumentation} site.

% espace avant Attention ou Note  :  mm
\vspacepdf{3mm}
\textbf{Note} : in this example file, the names of the payees etc are pure invention; any similarity with a real person or business is entirely accidental.

\section{Creation of a new set of accounts\label{start-newfile}}

The first time you use Grisbi, you will need to create a first
\indexword{accounts file}\index{accounts file}. The \gls{extension} of this file will be \file{.gsb} and its name will be \file{your-file-name.gsb}.

Immediately afterwards, you will need to create at least one account, and then some other accounts (current accounts, savings, credit, possibly a cash account and some transition accounts) that will contain their respective transactions.

For personal accounting, you will normally have only one account file, as this supports all the links between your different accounts. If you manage a busniess, or another personal account without a financial relationship with the first one, you will create another account file, which will have another name \file{your-second-file.gsb}. Thus \indexword{accounting entities}\index{accounting entity} will remain well separated.


% espace avant Attention ou Note  : 5 mm
\vspacepdf{5mm}

\strong{Caution}: for a given reporting entity, it is necessary and important to distinguish between the  \og \indexword{accounts file}\index{accounts file}  \fg{} and the  \og \indexword{account files}\index{account files} \fg{}:
\begin{itemize}

\item The \og accounts file  \fg{} you have created will have the extension \file{.gsb} and the name of \file{your-file.gsb}; it contains all the data of all the accounts created for the management of an accounting entity;

\item The \og account files  \fg{} are files that you may need to use or create to import or export data from one accounting application to another; these files will only contain data from one account (current or otherwise) at a time; they will have different extensions (\file{.ofx}, \file{.csv} or \file{.qif}) depending on their content; for more details, see the chapter \vref{move}, \menu{Export and import of accounts}.
\end{itemize}

% espace aprs Attention ou Note  : 5 mm
\vspacepdf{5mm}
In other words, all the accounts in your household accounts are recorded in a single accounts file, and all the accounts in your business are stored in a separate accounts file; and an account in Grisbi can correspond to an account file, but only when talking about importing or exporting data.

% espace pour changement de thme
\vspacepdf{5mm}

The general procedure for creating an account file is as follows: click on the menu File - New Account File; the account file creation wizard opens, which includes six steps. In the sixth step, the assistant offers you:

\begin{itemize}
\item  create a new account, and then follow the account creation wizard, which itself includes five steps, to create the first account (because it is essential to have at least one account);
\item or to use pre existing data, then use the import wizard, which also includes five steps, to import existing account operations.
\end{itemize}

After creating this first account or importing existing account transactions, if you want to create other accounts, you will return to the end of the process of creating the account file, which will return you in both cases to the create a new account option.


% espace pour changement de thme
\vspacepdf{5mm}
To create your accounts file, click the File - New Account File menu; the detailed procedure is as follows:
enumerate
welcome window: confirm with the Forward button;

To create your accounts file, click the \menu{File - New Account File} menu ; the detailed procedure is as follows:

\begin{enumerate}
\item New file assistant welcome window: confirm with the \menu{Forward} button ;
\item General configuration
\ifIllustration générale\refimage{start-file-create-img} :
\else générale :
\fi

\ifIllustration
% image centre
\begin{figure}[htbp]
\begin{center}
\includegraphics[scale=0.5]{image/screenshot/start_file_create}
\end{center}
\caption{General configuration of an accounts file}
\label{start-file-create-img}
\end{figure}
% image centre
\fi

\begin{enumerate} 
 \item choose the name of the accounting entity whose accounts you are managing, for example \og My accounts \fg{}, which can be chosen as the title of the Grisbi application home page,
\item enter the name of the accounts file with its complete tree; Grisbi defaults to the same name as the reporting entity, but you can change it
\item check the  \menu{Encrypt Grisbi} box if you wish to \gls{encrypt} the accounts file,
\item select the \indexword{date format}\index{date format} with one of the two buttons: dd/mm/yyyy for day/month/year, or  mm/dd/yyyy for month/day/year,
\item choose the decimal \indexword{separator}\index{separator} and the thousands from the drop-down lists,
 \item fill in the address (optional),
 \item  confirm with the  \menu{Forward} button;
\end{enumerate}

\item selection of the base \indexword{currency}\index{currency} :
\begin{enumerate} 
 \item click on the chosen currency in the list,
\item check the "include obsolete currencies" box if you also want to display old currencies,
\item confirm with the \menu{Forward} button;
\end{enumerate}

\item selection of the  \indexword{list of categories}\index{catgories !types} you will use :
\begin{enumerate} 
 \item click on your desired category, either the \menu{Standard category list} or the \menu{Empty list} \footnote{\strong{Translators Note:} Users installing the program on a system with a French Language interface will find different categories are offered including some for business users} 
\item check the \menu{Display foreign category sets} box to check if other categories are available\footnote{\strong{Translators Note:} This option is mainly for the benefit of users of a computer system with the French Language interface who will then be shown the two English categories mentioned in the previous step}
\item confirm with the \menu{Forward} button;
\end{enumerate}

\item Enter details of \indexword{banks}\index{banques !définition} holding your accounts :
\begin{enumerate} 
 \item click  \menu{Add} to define a bank; fill in the details of the bank (name, bank code, etc.), then confirm with the \menu{Add} to add the bank,
\item select a bank from the list and click the \menu{Remove} button to delete a bank, then confirm in the window that opens,
\item repeat actions a and b as many times as necessary,
\item  confirm with the \menu{Forward}  button to go to the next step, \menu{Creating a new account} ;
\end{enumerate} 

\item configuration completed: the configuration of the accounts file is complete, and this window offers you to choose one of the two methods of creating your first account
\ifIllustration compte\refimage{start-account-choice-img} :
\else compte :
\fi

\ifIllustration
% image centre
\begin{figure}[ht]
\begin{center}
\includegraphics[scale=0.5]{image/screenshot/start_account_choice}
\end{center}
\caption{Selecting the first account}
\label{start-account-choice-img}
\end{figure}
% image centre
\fi

\begin{itemize}
\item \menu{Create a new empty account} : if you check this line, then if you confirm with the \menu{Close} button, this window closes and the new account creation wizard starts. See  \vref{accounts-new}, \menu{Creating a new account},  which fully describes this procedure, then return to this page ;

\item \menu{From data from a bank file or other software} : if you check this line and then confirm with the \menu{Close} button ,  this window closes and the Import Data Wizard of a file account by Grisbi starts. See the \vref{move-import-importinit} section, \menu{Importing Account Files from Another Programme into Grisbi}, which fully describes this procedure, then return to this page.
\end{itemize}
\end{enumerate}

% tiquette du paragraphe suivant, pour que les liens hypertexte dans account.tex et QIF.tex  arrivent bien dessus
\label{start-newfile-end}

\textit{\textbf{In one way or another}}, you have now created your accounts file, as well as the first account of this file.

%espace pour changement de thme
\vspacepdf{5mm}

If you want to create other accounts now, select the \menu{Edit - New Account} to create another account (see the \vref{accounts-new}, \menu{ Creating a new account} section).

%espace pour changement de thme
\vspacepdf{5mm}

Otherwise, you can start using the account you just created or the one from which you just imported the data.

% espace avant Attention ou Note  : 5 mm
\vspacepdf{5mm}

\strong{Warning} : in general, it is inadvisable to have accents or spaces in the names of directories and files used by Grisbi. If so, rename them now. For example, spaces can be replaced by underscores (\underline{ }).

% saut de page pour titre solidaire
\newpage


\section{Saving your accounts file\label{start-save}}

Your operations are not written as you enter them as they might be in other software; you must therefore save your account file before exiting. Do not worry, Grisbi warns you if you have not done so.

You can configure the options for saving the account file in the  \menu{ Edit - Preferences} menu, see the section \vref{setup-general-files-manage}, \menu{Managing Account Files.}.


\section{Import from other personal accounting software}

See the \vref{move-import-importinit} section to import account files from another program into Grisbi. For the moment, Grisbi supports \gls{Gnucash}, \gls{OFX}, \GLS{CSV} and \GLS{QIF} formats.


